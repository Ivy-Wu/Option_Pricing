\documentclass{article}
\usepackage{geometry}
\geometry{portrait, margin=1in}
\usepackage[utf8]{inputenc}
\usepackage{amsmath,amsfonts,amsthm} % Math
\usepackage{amsfonts}
\usepackage{tikz}
\usepackage{graphicx} % takes care of graphic including machinery
\usepackage{tabularx} 
\usepackage{tikz} 
\usetikzlibrary{matrix,positioning}
\setlength{\parindent}{0em}
\tikzset{bullet/.style={circle,fill,inner sep=2pt}}



\begin{document}
\title{Option Pricing 101}
\date{}
\maketitle


\section{One-Period Binomial Tree}
We start will the simplest one-period binomial tree model, which assumes the stock price can take on two values in one period of time. Assume that the stock is currently trading at \$100 and the two possible outcome for the next period is \$120 and \$80 with equal probability. We have a one-period call option with strike price \$100. We also assume the interest rate is 0\% for simplicity.
\begin{center}
    

\begin{tikzpicture}[>=stealth,sloped]
    \matrix (tree) [%
      matrix of nodes,
      minimum size=1cm,
      column sep=3.5cm,
      row sep=1cm,nodes={text width=9em}
          ]
    {
          &   &  \\
          & {Stock price =\$110\\ Option Payoff=\$10} &   \\
     Stock price = \$100 &   &  \\
          &  {Stock price =\$85\\ Option Payoff =\$0} &   \\
          &   &  \\
    };
    \node[bullet,right=0mm of tree-3-1.east](b-3-1){};
    \node[bullet,left=0mm of tree-2-2.west](b-2-2){};
    \node[bullet,left=0mm of tree-4-2.west](b-4-2){};
    \draw[->] (b-3-1) -- (b-2-2) node [midway,above] {$p$};
    \draw[->] (b-3-1) -- (b-4-2) node [midway,below] {$(1-p)$};
  \end{tikzpicture}
\end{center}

It is very tempting to conclude that the option price is $$0.5\cdot 10 + 0.5 \cdot 0 = 5$$ given the probability and the option payoff in both scenarios. We defer the justification of why this calculation is incorrect after the discussion of the following portfolio.\\

We are interested in constructing a portfolio of stocks and bonds(face value \$100) to replicate the payoff of a single call option. Is that possible? Suppose we want $x$ share of stocks and $y$ shares of bonds. The portfolio needs to satisfy the following constraints:
\begin{align*} 
110x + 100y &=  10 \\ 
85x + 100y &=  0
\end{align*}
Solving the system, we get $x = 0.4$ and $y = -0.34$, which means by longing $0.4$ share of the stock and shorting $0.34$ share of bond, we successfully construct a replicating portfolio which \textbf{replicates} the payoff of a call option. By no-arbitrage law, the set-up cost of the replicating portfolio should be exactly the same as the price of a call option. The portfolio costs $0.4\cdot 100 - 0.34*100 = 6$ to set up.\\

The question is, why does the call option only cost \$6 in which the expected payoff is actually \$10. The intuitive explanation behind this is that since the price movement of the underlying security is asymmetric, i.e. downside is bigger than the upside (from a long stock perspective, the expected payoff is $0.5\cdot 10 + 0.5\cdot (-15) = -2.5$). The option buyer would request to pay a lower price than the expected payoff from the option position.\\

\textbf{Exercises}
\begin{enumerate}
    \item What would be the expected payoff and the option price if the stock price will be \$120 and \$ 80 respectively? What about \$120 and \$90? What can you conclude by comparing the three cases?
\end{enumerate}

\section{Risk Neutral Probabilities}
Since the option price can be different from the expected payoff from the option position, a new probability measure is introduced - \textbf{Risk-Neutral Probability}. In our previous example, the risk neutral probability is the probability measure that equates the expected payoff and the option price.\\

To find the risk-neutral probability, we consider a more generic case where the two possible outcomes at $T=1$ are $uS$ and $dS$ where $S$ is the current stock price and $u$ and $d$ are up/down factor respectively. We denote the risk-neutral up probability to be $\hat{p}$.\\

First, we need to find the option price through the replicating portfolio.
\begin{align*}
    xuS + yB_T &= (u-1)S\\
    xdS + yB_T &= 0
\end{align*}
Solving the system above, we get $x = \frac{(u-1)}{(u-d)}$ and $y = -
\frac{(u-1)dS}{(u-d)B_T}$. Hence the set-up cost of the replicating portfolio is $$xS + y B_0 = \frac{u-1}{u-d}S - \frac{(u-1)dS}{(u-d)(1+r)} = \frac{(u-1) (1+r-d)S}{(u-d)(1+r)}$$

Now we can solve for the risk neutral probabilities by equating the future value of the option with the expected payoff. $$\hat{p}(u-1)S + (1-\hat{p})0 = (1+r)\cdot \frac{(u-1) (1+r-d)S}{(u-d)(1+r)}$$
and we get $\hat{p} = \frac{1+r-d}{u-d}$. Note that there is not direct relationship between the actual probabilities and the risk-neutral probabilities.
\section{Black-Scholes Equation}
\

\end{document}
